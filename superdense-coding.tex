\documentclass[a4paper,12pt]{amsbook}
\usepackage{physics}
\usepackage{amsmath,amssymb,amsthm}
\usepackage{mathtools}
\usepackage{mathpazo}
\usepackage{microtype}
\providecommand{\tb}{\textbackslash}
\newcommand{\0}{{$|0\rangle$}}
\newcommand{\1}{{$|1\rangle$}}
\newcommand{\bellstates}{{$\frac{|00\rangle + |11\rangle}{\sqrt{2}}, 
\frac{|00\rangle - |11\rangle}{\sqrt{2}}, 
\frac{|01\rangle + |10\rangle}{\sqrt{2}},
\frac{|01\rangle - |10\rangle}{\sqrt{2}}$
}}
\usepackage{graphicx}
\usepackage{amsfonts}
\usepackage[utf8]{inputenc}
\usepackage[T1]{fontenc}
\usepackage{geometry}
\geometry{a4paper, margin=1in}
\usepackage{listings}
\usepackage{xcolor}
\pagecolor{black}
\color{white}
\usepackage{parskip}
\setlength{\parindent}{0pt}

\begin{document}

\title{Superdense Coding}
\author{Vivek Soorya Maadoori}
\date{\today}
\maketitle
Superdesnse coding is about transformations

\section{setup}
\begin{enumerate}
    \item \textbf{Task}: Send 2 classical bits by sending 1 qubit across

\item To carry out this task we use an entangled pair and transform it to encode information;

\item There are four entangled pairs:

\bellstates,

and the possibilities of four values of 2 classical bits:

00, 11, 01, 10;

00: $\frac{|00\rangle + |11\rangle}{\sqrt{2}}$,
11: $\frac{|00\rangle - |11\rangle}{\sqrt{2}}$,
01: $\frac{|01\rangle + |10\rangle}{\sqrt{2}}$,
10: $\frac{|01\rangle - |10\rangle}{\sqrt{2}}$

\item The four entangles pairs are I, X, Z and XZ transformations apart \footnotemark[2] 

    \footnotetext[2]{XZ = -2iY, but -2 is global amplitude scalar, and this does not affect our bell state to classical doublt-bit encoding; the phase is global and so can be ignored and the  magnitude disappears upon renormalization}\footnotemark[1]
    \footnotetext[1]{permutational invariance for bell states and transformations}

\item Using these transformations Alice can send her two classical bits to Bob by sending over a tranformed qubit. Upon receipt of the qubit, Bob can measure it in the Bell basis\footnotemark[3] since it forms an orthonormal basis \footnotemark[4].


\footnotetext[3]{Appendix C}
\footnotetext[4]{Refer to appendix B, to see how bell basis forms an orthonormal basis.}

\end{enumerate}

\section{solution}
steps:
\begin{enumerate}
    \item Alice prepares two qubits in an EPR state, and sends one of the qubits to Bob
    \item Bob performs the necessary transformation to reflect his choice and sends it back
    \item Alice measures the two qubits in bell basis and uses the map to obtain the classical information that is Bob's choice.
\end{enumerate}



\section{Appendix A}
\begin{enumerate}
    \item When there are 32 superpositions, why are the 4 states so special that they get their own two names (EPR pairs and bell states)? 

    \item There are 32 equally weighted superpositions of $|10\rangle, |01\rangle, |10\rangle, |11\rangle.$
Square the cardinality of the set of equally weighted superpositions. Taking into account negative relative phase, we double that number.
    $\Rightarrow 2 \times 4^2 = 32.$ 

    \item Of these 8 are not actually superpositions\footnotemark[5]:

        \begin{itemize}
    \item     $\frac{|00\rangle + |00\rangle}{2},\frac{ |00\rangle - |00\rangle}{2},\frac{ |01\rangle + |01\rangle}{2}, \frac{ |01\rangle - |01\rangle}{2}, $

    \item     $\frac{ |10\rangle + |10\rangle}{2},\frac{ |10\rangle - |10\rangle}{2},\frac{ |11\rangle + |11\rangle}{2},\frac{ |11\rangle - |11\rangle}{2}, $
        \end{itemize}

    \item Of the remaining 12 are duplicates owing to commutativity of addition:

        $\{\{\frac{1}{\sqrt{2}}(|00\rangle + |10\rangle),  \frac{1}{\sqrt{2}}(|10\rangle + |00\rangle)\}, \dots\}$

    where each pair differ at most by global phase like in

    $\frac{1}{\sqrt{2}}(|00\rangle - |10\rangle)$ \text{ and }$\frac{1}{\sqrt{2}}(|10\rangle - |00\rangle)$

    \item Of the remaining, 8 are not entangled, meaning they are tensor product decomposable.

    \item This leaves us with $2\times4^2 - 12 - 8 = 32 - 20 = 4$ which are

        \begin{itemize}
    \item $\frac{|100\rangle + |110\rangle}{\sqrt{2}}$ 
    \item $\frac{|100\rangle - |111\rangle}{\sqrt{2}}$ 
    \item $\frac{|101\rangle + |111\rangle}{\sqrt{2}}$
    \item $\frac{|101\rangle - |110\rangle}{\sqrt{2}}$
\end{itemize}

\footnotetext[5]{consider $\frac{|00\rangle - |00\rangle}{2}$, what would phase mean here? especially since differing by relative phase does not yield orthogonality in this case. That is, when $\frac{|00\rangle - |00\rangle}{2} = \frac{|00\rangle + |00\rangle}{2}$ what does relative phase signify then?}

\end{enumerate}


\section{Appendix B}
\begin{enumerate}
    \item Bell states are pair wise orthogonal\footnotemark[6]a
$\frac{|00\rangle + |11\rangle}{\sqrt{2}} \text{ and }
\frac{|00\rangle - |11\rangle}{\sqrt{2}}$ 
are orthogonal since they differ by relative phase

    \item Same with 
$\frac{|01\rangle + |10\rangle}{\sqrt{2}} \text{ and }
\frac{|01\rangle - |10\rangle}{\sqrt{2}}$

    \item Now consider 
$\frac{|00\rangle + |11\rangle}{\sqrt{2}} \text{ and }
\frac{|01\rangle - |10\rangle}{\sqrt{2}}$

    \item These two are also orthogonal since,
$(\frac{\langle00|+ \langle11|}{\sqrt{2}}) 
(\frac{|01\rangle - |10\rangle}{\sqrt{2}})$

$ = (\frac{\langle00|01\rangle - \langle00|10\rangle +  \langle11|01\rangle - \langle11|10\rangle}{\sqrt{2}})$

$ = (\frac{\langle0|0\rangle \cdot \langle0|1\rangle - \langle0|1\rangle \cdot \langle0|0\rangle + \langle1|0\rangle \cdot \langle1|1\rangle  - \langle1|1\rangle \cdot \langle1|0\rangle }{2}) $

$ = \frac{1\cdot0 - 0\cdot1 + 0\cdot1 - 1\cdot0}{2}$

$ = \frac{0 - 0 + 0 - 0}{2}$

$= 0$

    \item By transitivity all the bell states are orthogonal to each other
\footnotetext[6]{Where inner product is defined as product of inner product of qubits in the respective subsystems ie constituent Hilbert spaces of the tensor product hilbert space}

\end{enumerate}


\section{Appendix C}
\begin{enumerate}
    \item You have to measure it in the bell basis and not any other basis like the computational basis, since measuring it in other bases would not return any bell state with probability 1, but when measured int he bell basis, one of the four bell states is returned with probability 1 and the other three with probability 0, accomodating determinstic communication.

    \item Consider the following scenario:
\begin{enumerate}
    \item If you are Bob and trying to decode the information Alice sent (assuming you have the mapping from qubit state to classical information), you might be tempted to measure it in the computational basis, but you will quickly (or taking your time) realize that you will lose information about relative phase and will only be able to pin the information down to one of two camps of the bell states, where members of the camp only differ by a relative phase.
   \item  Note then that at this point you want to extract phase information.
   \item  Also note that if you want to extract any information at all, you will have to measure, or do it before measurement.
   \item  All we can done before the measurement, is unitary transformations and so to recover/extract the relative phase, perform a unitary transformation. It turns out that \textit{the} unitary transformation (for there could be any number of unitary transformations) that does the job is basis change transformation. Or you could just shorten the process and directly measure it in the bell basis. But if you want to measure it in the computational basis, first perform the necessary unitary transformation.
\end{enumerate}
\end{enumerate}


\end{document}
