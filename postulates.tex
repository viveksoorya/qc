\documentclass{article}
\usepackage{physics}
\usepackage{amsmath}
\usepackage{amssymb}
\providecommand{\tb}{\textbackslash}
\newcommand{\0}{{$|0\rangle$}}
\newcommand{\1}{{$|1\rangle$}}
\usepackage{graphicx}
\usepackage{amsfonts}
\usepackage[utf8]{inputenc}
\usepackage[T1]{fontenc}
\usepackage{geometry}
\geometry{a4paper, margin=1in}
\usepackage{listings}
\usepackage{xcolor}
\pagecolor{black}
\color{white}
\usepackage{parskip}
\setlength{\parindent}{0pt}

\begin{document}

Unitary evolution is essentially basis change. When not measured, the quantum system changes its basis continuously. When measured, the quantum system fixes its basis. 

After measurement, the quantum system continues to exist in the new basis, but since we are not measuring it anymore, shouldn't it keep changing its basis; 

It feels like the Evolution postulate is contending with the Measurement postulate. Contradiction arises from the faulty assumption that the all unitary transformations are merely basis changes.


Odd sized matrices cannot exist in quantum computation, since operators are acting on vectors of size $2^n$. Moreover, the size of the operator is also $2^n \times 2^n$.

Are the Pauli matrices the only Hermitian matrices that are also unitary?
\end{document}
