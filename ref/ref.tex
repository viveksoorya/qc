\documentclass{article}
\usepackage[utf8]{inputenc}
\usepackage{amsmath, amssymb}
\usepackage{graphicx}
\usepackage{hyperref}
\usepackage{geometry}
\usepackage{enumitem}

\begin{document}

\subsection*{Proof cycle and related stuff}

How do I prove this? How do I approach thinking about this in terms of proof; How do I go from this conjecture to thinking; we need to start from things we know to be true and arrive here; this method is known as direct proof; there are other methods too, but a proof generally needs an idea;
\begin{enumerate}
    \item Understand every term of the proof at a cursory level.
    \item Understand what is being asked in those terms?
    \item Now you understand the task at this point.
    \item Ideate and attempt; sketch out promising avenues that look hopeful.
    \item Once you have a promising sketch, write it in better detail.
    \item Cycle until you reach a satisfactory proof.
    \item Write down ideas while Ideating. Think on paper.
\end{enumerate}

\subsubsection{Pauli gates on $e_i$}

\subsection{Pauli Gates}
Quantum gates are simply operators. But since we are doing quantum computation, we adapt the term gate instead of operators.

\[
I = \begin{bmatrix} 1 & 0 \\ 0 & 1 \end{bmatrix} \quad
X = \begin{bmatrix} 0 & 1 \\ 1 & 0 \end{bmatrix} \quad
Y = \begin{bmatrix} 0 & -i \\ i & 0 \end{bmatrix} \quad
Z = \begin{bmatrix} 1 & 0 \\ 0 & -1 \end{bmatrix}
\]
\begin{center}
  \begin{tabular}{cccc}
I & X & Y & Z \\
  & bit flip    &   & phase flip \\
  & quantum not &   &            \\
\end{tabular}
\end{center}

\begin{align*}
I |0\rangle &= |0\rangle \\
I |1\rangle &= |1\rangle \\
X |0\rangle &= |1\rangle \\
X |1\rangle &= |0\rangle \\
Y |0\rangle &= -i |1\rangle \\
Y |1\rangle &= i |0\rangle \\
Z |0\rangle &= |0\rangle \\
Z |1\rangle &= -|1\rangle \\
\end{align*}



\end{document}
