\documentclass{article}
\usepackage[utf8]{inputenc}
\usepackage{amsmath, amssymb}
\usepackage{graphicx}
\usepackage{hyperref}
\usepackage{geometry}
\usepackage{enumitem}

\begin{document}

\subsection{definitions}
norm preserving transformations are called isometries; (metron [Greek] meant measure)

\subsection{notation}
$\mathcal{L}$(V,W): set of linear map $\mathcal{L}$ from vector spaces V to W

\subsection*{Proof cycle and related stuff}

How do I prove this? How do I approach thinking about this in terms of proof; How do I go from this conjecture to thinking; we need to start from things we know to be true and arrive here; this method is known as direct proof; there are other methods too, but a proof generally needs an idea;
\begin{enumerate}
    \item Understand every term of the proof at a cursory level.
    \item Understand what is being asked in those terms?
    \item Now you understand the task at this point.
    \item Ideate and attempt; sketch out promising avenues that look hopeful.
    \item Once you have a promising sketch, write it in better detail.
    \item Cycle until you reach a satisfactory proof.
    \item Write down ideas while Ideating. Think on paper.
\end{enumerate}

\subsubsection{Pauli gates on $e_i$}

\subsection{Pauli Gates}
Quantum gates are simply operators. But since we are doing quantum computation, we adapt the term gate instead of operators.

\[
I = \begin{bmatrix} 1 & 0 \\ 0 & 1 \end{bmatrix} \quad
X = \begin{bmatrix} 0 & 1 \\ 1 & 0 \end{bmatrix} \quad
Y = \begin{bmatrix} 0 & -i \\ i & 0 \end{bmatrix} \quad
Z = \begin{bmatrix} 1 & 0 \\ 0 & -1 \end{bmatrix}
\]
\begin{center}
  \begin{tabular}{cccc}
I & X & Y & Z \\
  & bit flip    &   & phase flip \\
  & quantum not &   &            \\
\end{tabular}
\end{center}


\section{come back to this later}
what is the POVM formalism? 
\begin{enumerate}
  \item You use it when you don't care about the post-measurement state. 
  \item It is an established elegant formation that is adopted by researchers in quantum computation and quantum information.
  \item What is the added advantage of a POVM? What exactly is elegant about the formalism?
\item How exactly is the POVM formation used for the analysis of measurements? 
$\{M_m\}$ satisfy the completeness relation. What else is necessary to go from operator to measurement operator. Does it say, D a) have to be hermitian? 


\end{enumerate}



\end{document}
