\documentclass{article}
\usepackage{amsmath} % Required for align* environment and bmatrix
\usepackage{graphicx} % Required for inserting images
\usepackage{amssymb} % Required for blacksquare (QED)
\usepackage{geometry}
 \geometry{
 a4paper,
 total={170mm,257mm},
 left=20mm,
 top=20mm,
 }
\usepackage{braket}
\usepackage{xcolor}
%\pagecolor{black}    % Sets the page background to black
%\color{white}        % Sets the text color to white

\title{quantum computing and quantum information}
\author{Vivek Soorya}
\date{July 2025}

\begin{document}

\maketitle
\section{at a glance}
Exercises 1, and 4 are too trivial; consider them solved;
Exercise 2 is solved
Exercixes 3 and 5 are incomplete





\section{lemmas to prove and questions to answer}
\begin{enumerate}
    \item Show that
    \begin{enumerate}
        \item Show that \\
        $\langle\psi|M|\psi\rangle = \langle\psi|P_m|\psi\rangle = P(m) \text{ for } P_m $ 
        fixed  
        \item Show that Given a transformation in finite dimensions from a vector space to itself, and the input basis $e_i$, the matrix columns are the output bases are the matrix columns.
        \item SHow that given a matrix transformation in terms of input bases $e_i$, and output bases; the output bases become the matrix's columns. How can I systematically explore the prove for this
        \item{Exercise 2.5: Inner Product on $\mathbb{C}^n$}
    We need to show that $\langle \phi | \psi \rangle = \sum_{i=1}^n \phi_i^* \psi_i$ is a valid inner product on $\mathbb{C}^n$. An inner product must satisfy three properties:
    \begin{enumerate}
        \item \textbf{Conjugate symmetry}: $\langle \phi | \psi \rangle = \langle \psi | \phi \rangle^*$.
        \item \textbf{Linearity in the second argument}: $\langle \phi | \alpha \psi + \beta \chi \rangle = \alpha \langle \phi | \psi \rangle + \beta \langle \phi | \chi \rangle$.
        \item \textbf{Positive-definiteness}: $\langle \phi | \phi \rangle \ge 0$, and $\langle \phi | \phi \rangle = 0$ if and only if $\ket{\phi}$ is the zero vector.
    \end{enumerate}
    Let $\ket{\phi} = \begin{pmatrix} \phi_1 \\ \vdots \\ \phi_n \end{pmatrix}$ and $\ket{\psi} = \begin{pmatrix} \psi_1 \\ \vdots \\ \psi_n \end{pmatrix}$.

    \end{enumerate}
\item answer these
    \begin{enumerate}
        \item What kind of transformation results when a matrix is transformed by its adjoint. 
$$\left[ \begin{array}{cc} a & c \\ b & d \end{array} \right] \left[ \begin{array}{cc} a & b \\ c & d \end{array} \right] = \left[ \begin{array}{cc} a^2+c^2 & ab+cd \\ ab+cd & b^2+d^2 \end{array} \right] $$ 
        \item Is $$ \langle\psi|M|\psi\rangle = ? \langle\psi|P_m|\psi\rangle$$
             We know that, \\
             $$\langle\psi|M|\psi\rangle = \sum_m \langle\psi|P_m|\psi\rangle $$
            Now, for fixed $ m \equiv m_a $ every projector that is not $P_m$ will take $\langle\psi|P_m|\psi\rangle$ to the kernel, gives 0 as the expectation value for that projection, leaving $\langle\psi|P_{m_a}|\psi\rangle$ to result. Thus, for fixed $ m \equiv m_a $, \\
            $ \langle\psi|M|\psi\rangle = \langle\psi|P_{m_a}|\psi\rangle \\\blacksquare$
        \item Can a scalar have a matrix representation, can a vector have a matrix representation
        
        \item Is the outer product like a projection onto the vector space itself, with the vector that outer product vector being a sort of light source and the vector being operated on being the object \& the resultant vector being the projection? It is essentially an additional left multiplication with the dual.
        \item For quantum mechanics, we only need the vector space made of vectors whose norm is $\le 1$. Can we discard the rest of the vector space? (i.e. where the norm is $> 1$)
        \item Given an isolated physical system, what is the state space?
        \item Is negative like eigenvalue interpreted as phase?
        \item What is a determinant, physically? Cuz they are used in characteristic equation to extract eigenvalues and eigenvectors

    \end{enumerate}

    \end{enumerate}
    


\section{never gonna practice}

\subsection*{working on these}
\begin{enumerate}
    
    \item A + B + C = inner product where 
        \subitem A: $ \langle v_1 | v_2 \rangle = \langle v_2 | v_1 \rangle^*$ 
        \subitem B: $ \langle v_1 | v_1 \rangle \ge 0 $ and is equal to 0 iff $ v_1 = 0 $
        \subitem B: projection of a vector with itself are always positive complex values.
        \subitem C: $ \langle v_1 | (\lambda v_2)\rangle = \lambda \langle v_1 | v_2 \rangle = $ linearity 

\end{enumerate}

    

\end{document}